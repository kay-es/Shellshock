\section{Maßnahmen gegen SQL-Injections}

\begin{frame}{SQL-Injections}
\begin{block}{Generelles}
\begin{itemize}
	\item Stellen seit Einführung von Datenbanken ein Sicherheitsrisiko dar
	\item Machen ca. 69\% aller Angriffe aus \footnotemark
	\item Benutzereingaben werden in der SQL-Abfrage missinterpretiert
\end{itemize}
\end{block}
\setbeamerfont{footnote}{size=\tiny}
\footnotetext{Stand: Q2 2012; Quelle: http://www.zdnet.com/article/sql-injection-attacks-up-69/}
\setbeamerfont{footnote}{size=\footnotesize}
\end{frame}

\begin{frame}{Klassische Angriffe}
\begin{block}{Numerische SQL-Abfrage}
	SELECT * FROM Users WHERE ID = \textcolor{red}{\textbf{1 OR 1=1}}
\end{block}
\begin{block}{SQL-Abfrage mit String-Wert}
	SELECT * FROM Users WHERE name = '{}\textcolor{red}{\textbf{nickname'{} OR 1=1- -}}'{}
\end{block}
\end{frame}


\begin{frame}{Maßnahmen gegen SQL-Injections}
\begin{block}{Lösungsansätze}
\begin{itemize}
\item Blacklisting 
\item Whitelisting
\item Escaping
\item Prepared Statements \& Bind Variablen
\end{itemize}
\end{block}
\end{frame}

\begin{frame}{Blacklisting}
\begin{block}{Vorgehen}
\begin{itemize}
\item Benutzereingaben werden gefiltert
	\begin{itemize}
	\item Metazeichen von der Eingabe entfernen (z.B.: ", ;, ', -, \%)
	\item Getrimmter Substring wird im SQL-Query verwendet
	\end{itemize}
\end{itemize}
\end{block}
\begin{block}{Beispiel}	
	remove\_metachar("{}\textcolor{red}{\textbf{nickname'{} OR 1=1- -}}{}") $\Rightarrow$ "{}\textcolor{red}{\textbf{nicknameOR1=1}}{}" \\
	$\Rightarrow$ SELECT * FROM Users WHERE name = '{}\textcolor{red}{\textbf{nicknameOR1=1}}'{}

\end{block}
\end{frame}

\begin{frame}{Blacklisting}
\begin{block}{Vorgehen}
\begin{itemize}
\item Benutzereingaben werden gefiltert
	\begin{itemize}
	\item Metazeichen von der Eingabe entfernen (z.B.: ", ;, ', -, \%)
	\item Getrimmter Substring wird im SQL-Query verwendet
	\end{itemize}
\end{itemize}
\end{block}
\begin{block}{Problematik \& Sicherheit}	
\begin{itemize}
\item Vergessen von einem einzigen Metacharakter kann kompletten Lösungsansatz zerstören
\end{itemize}
\end{block}
\end{frame}


\begin{frame}{Whitelisting}
\begin{block}{Vorgehen}
\begin{itemize}
	\item Nur gültige Zeichen zulassen
	\item Validierung \& Überprüfung mittels regulärem Ausdruck
\end{itemize}	
\end{block}
\begin{block}{Beispiel}	
	Whitelist: [0,9]* \\
	check\_whitelist("{}\textcolor{red}{\textbf{1 OR 1=1}}{}") $\mapsto$ "{}\textcolor{red}{\textbf{ungültige Eingabe}}{}" \\
	trim\_whitelist("{}\textcolor{red}{\textbf{1 OR 1=1}}{}") $\mapsto$ "{}\textcolor{red}{\textbf{ungültige Eingabe}}{}" \\
	$\Rightarrow$ SELECT * FROM Users WHERE ID = \textcolor{red}{\textbf{1}}
\end{block}
\end{frame}

\begin{frame}{Whitelisting}
\begin{block}{Vorgehen}
\begin{itemize}
	\item Nur gültige Zeichen zulassen
	\item Validierung \& Überprüfung mittels regulärem Ausdruck
\end{itemize}	
\end{block}
\begin{block}{Problematik \& Sicherheit}	
\begin{itemize}
\item Herr O'Neil kann sich nicht mehr mit seinem Namen registrieren
\end{itemize}
\end{block}
\end{frame}


\begin{frame}{Escaping}
\begin{block}{Vorgehen}
\begin{itemize}
\item Benutzereingabe wird so umgeformt, dass die Datenbank sie richtig interpretieren kann
\item Funktioniert nur bei String-Eingaben
\end{itemize}
\end{block}
\begin{block}{Beispiel}
	escape("{}\textcolor{red}{\textbf{O'Neil}}"{}) $\mapsto$ "{}\textcolor{red}{\textbf{O"{}Neil}}"{} \\
	$\Rightarrow$ Datenbank interpretiert "{} als einfaches ' und nicht mehr als Beginn bzw. Ende eines Strings
\end{block}
\end{frame}

\begin{frame}{Escaping}
\begin{block}{Vorgehen}
\begin{itemize}
\item Benutzereingabe wird so umgeformt, dass die Datenbank sie richtig interpretieren kann
\item Funktioniert nur bei String-Eingaben
\end{itemize}
\end{block}
\begin{block}{Problematik \& Sicherheit}
\begin{itemize}
\item escape()-Funktion muss konsistent und überall (bei INSERT, UPDATE, DELETE, etc.) angewendet werden
\item Ermöglicht sonst Second Order SQL-Injections
\end{itemize}
\end{block}
\end{frame}

\begin{frame}{Prepared Statements \& Bind Variablen}
\begin{block}{Vorgehen}
\begin{itemize}
\item Nutzen von Platzhaltern (z.B.: ?, :name, etc.) anstatt Eingaben direkt in der Datenbank-Abfrage zu verwenden
\item Query wird zuerst vorbereitet und danach erst die Parameter eingesetzt
\end{itemize}
\end{block}
\begin{block}{Beispiel}
prep\_stat = "{}SELECT * FROM Users WHERE ID = \textcolor{red}{\textbf{?}} AND name = \textcolor{red}{\textbf{?}}"{}; \\
prep\_stat.setPara(1, "{}\textcolor{red}{\textbf{1 OR 1=1}}{}"); \\
prep\_stat.setPara(2, "{}\textcolor{red}{\textbf{nickname'{} OR 1=1- -}}{}"); \\
prep\_stat.exec\_query;
\end{block}
\end{frame}

\begin{frame}{Prepared Statements \& Bind Variablen}
\begin{block}{Vorgehen}
\begin{itemize}
\item Nutzen von Platzhaltern (z.B.: ?, :name, etc.) anstatt Eingaben direkt in der Datenbank-Abfrage zu verwenden
\item Query wird zuerst vorbereitet und danach erst die Parameter eingesetzt
\end{itemize}
\end{block}
\begin{block}{Problematik \& Sicherheit}
\begin{itemize}
	\item Durch Platzhalter kann der Angreifer die SQL-Abfrage nicht verändern
	\item Die SQL-Abfrage kann nicht missinterpretiert werden
	\item Fast unmöglich die Abfrage zu manipulieren
	\item Bester Ansatz, um SQL-Injections zu vermeiden
\end{itemize}
\end{block}
\end{frame}

