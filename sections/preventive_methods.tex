% easier quotes
\newcommand{\gqq}[1]{\glqq#1\grqq}

\section{Präventive Verteidigungsmaßnahmen}

\begin{frame}{Präventive Verteidigungsmaßnahmen}
        \begin{exampleblock}{Mehrschichtige Sicherheitsstrukturen}
                \begin{itemize}
                        \item Experten: Sicherheitslücken nicht zu 100\% ausschließbar
                        \pause
                        \item Können aber durch mehrschichtige Sicherheitsstrukturen migriert werden
                        \pause
                        \item Torvalds: \gqq{The only real solution to security is to admit that bugs happen, and then mitigate them by having multiple layers, so if you have a hole in one component, the next layer will catch the issue.} 
                \end{itemize}
        \end{exampleblock}
\end{frame}

\begin{frame}{Präventive Verteidigungsmaßnahmen}
        \begin{exampleblock}{Zugriffskontrolle}
                \begin{itemize}
                        \item Zugriffskontrolle kann eine Schicht sein
                        \pause
                        \item Zentrale Frage: Wer darf auf welche Daten lesen, schreiben, ausführen?
                        \pause
                        \item Drei grundlegene Modelle:
                        \pause
                        \begin{itemize}
                                \item Mandatory Access Control (MAC)
                                \pause
                                \item Discretionary Access Control (DAC)
                                \pause
                                \item Role-based Access Control (RBAC)
                        \end{itemize}
                \end{itemize}
        \end{exampleblock}
\end{frame}

\begin{frame}{Präventive Verteidigungsmaßnahmen}
        \begin{exampleblock}{Sandboxing}
                \begin{itemize}
                        \item blablabla... 
                \end{itemize}
        \end{exampleblock}
\end{frame}

\begin{frame}{Präventive Verteidigungsmaßnahmen}
        \begin{exampleblock}{AppArmor}
                \begin{itemize}
                        \item Einfache Umsetzung von Mandatory Access Control (MAC)
                        \pause
                        \item Schränkt die Rechte von Applikationen ein
                        \pause
                        \item Im Vergleich zu SELinux einfache Konfiguration
                \end{itemize}
        \end{exampleblock}
\end{frame}

\begin{frame}{Präventive Verteidigungsmaßnahmen}
        \begin{exampleblock}{Docker}
                \begin{itemize}
                        \item blabla...
                \end{itemize}
        \end{exampleblock}
\end{frame}

\begin{frame}{Präventive Verteidigungsmaßnahmen}
        \begin{exampleblock}{Vergleich AppArmor \& Docker}
        \begin{itemize}
                \item blabla... 
        \end{itemize}
\end{exampleblock}
\end{frame}

\begin{frame}{Präventive Verteidigungsmaßnahmen}
        \begin{exampleblock}{Fazit}
                \begin{itemize}
                        \item blabla... 
                \end{itemize}
        \end{exampleblock}
\end{frame}

